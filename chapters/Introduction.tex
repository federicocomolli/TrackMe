\section{Purpose}
TrackMe is a company that wants to create a software-based system that allow third parties to monitor the location and health status of individuals. 
\par This service, which from now on will be referred to as Data4Help, supports the registration of individuals who, by registering, agree to the collection of their data. Data acquisition can occur through smart watches or similar electronic devices.
\par Third parties can request access to the data of some specific individuals of which they know a form of identification (such as the SSN or the Fiscal Code). In this case, TrackMe forwards the request to the user who can accept or refuse it.
\par Another service offered to third parties by Data4Help is to access to anonymous data of classes of individuals, grouped by different criteria (for example geographical area, age, sex, etc.). Anonymity is a crucial value for TrackMe, so it approves these requests only if third parties cannot go back to users' identity; for this reason requests referred to groups composed by less than 1000 individuals will be denied.
\par After a request has been accepted, third parties can subscribe to the reception of updated data as soon as they are produced.
\par Moreover, in order to exploit the features offered by Data4Help, TrackMe wants to develop a customized SOS service to elderly people. This system, called AutomatedSOS, uses the data collected by Data4Help to monitor the health status of the subscribed customers. Through the real time screening the system is able to react to the variation of the parameters by sending an ambulance to the location of the customer in case of emergency.

\pagebreak
\subsection{Goals}

Data4Help:  
\begin{enumerate} [label={[G\arabic*]}]
    \item Allow a visitor to become registered user after providing credentials.
    \item Monitor the location of users through electronic devices.
    \item Monitor the health status of users through electronic devices.
    \item Allow a user to accept or refuse a request to access to personal data.
    \item Allow third parties to register to the system.
    \item Allow third parties to request an access to users' data.
    \begin{enumerate} [label*={.\arabic*}]
        \item [{[G6.1]}] Allow third parties to request access to data of some specific individuals by providing SSN or Fiscal Code.
        \item [{[G6.2]}] Allow third parties to request access to anonymized data of group of people.
    \end{enumerate}
    \item Accept request for data of group of people only if the system can guarantee the anonymity of the people.
    \item Allow third parties to subscribe to new data and to receive them as soon as they are produced.
\end{enumerate}  
\noindent
AutomatedSOS:
\begin{enumerate} [resume, label={[G\arabic*]}]
    \item Allow a visitor to become registered user after providing credentials.
    \item Monitor the location of users through electronic devices.
    \item Monitor the health status of users through electronic devices.
    \item Send to the location of the customer very quickly an ambulance when such parameters are below certain thresholds. 
\end{enumerate}

\section{Scope}
\subsection{Analysis of the phenomena}
Data4Help application can be used after the registration through the interface on the electronic devices.
People who are for different reasons interested in services have to provide their name, surname, gender, birth date, SSN or Fiscal Code.
\par Users of Data4Help have to wear some kind of electronic devices that record information about their position and their health status.
Data4Help is an application-to-be that is projected to be installed on different wearable electronic devices (smart-watches, smart-bands, etc.) and eventually can interact with other health tools such as heart rate bands, smart scales or similar.
\par To identify the health status, Data4Help search for different parameters provided by sensors on smart-watches or health tools such as heart rate, skin temperature, blood glucose level, weight, etc.
Also, Data4Help acquires the position of its users through GPS technology.
\newline Data acquired by Data4Help may be matter of interest for third parties that can make a request using the functionality provided by the application. If a request is referred to a single user it's necessary to provide his/her SSN or Fiscal Code. In addition it's possible to request data referred to group of people (associated by age, gender, location, etc.). Data4Help has to accept or decline these requests using a defined privacy criteria.
\par Users who receive a request for their personal information can accept or decline it using the provided functionality.
\newline Data4Help offers the possibility to subscribe to data in order to receive the updated ones as soon they are produced.
\par Furthermore, TrackMe wants to build a new application on top of Data4Help, with the purpose of exploiting its functionality to provide an emergency service.
As for Data4Help, registration is necessary to use AutomatedSOS.
\newline AutomatedSOS analyzes real-time data provided by Data4Help technology.
If the software detects an anomaly in the parameters it notifies an emergency to the Operations Center of the National Health Service within 5 seconds. 
From now on the National Health Service is responsible for the first aid and it manages the request accordingly to its own protocol.
The intervention is carried on determining the emergency level and sending an ambulance to the location of the user.

\subsection{Stakeholders}
In this paragraph we analyze all the entities involved in the project of Data4Help and AutomatedSOS.
\par TrackMe is the main stakeholder of the project being that one who commissioned the project and will pay for it.
The core of the two applications are data provided by users that wants to monitor their health status.
Of course, this amount of data is a source of interest for third parties that wants to access it.
\par The National Health Service plays a fundamental role in the realization of AutomatedSOS taking care of the emergency intervention notified by the software-to-be.

\section{Definitions, Acronyms, Abbreviations}
\subsection{Definitions}
\begin{itemize}
    \item Users: people who use the services provided by TrackMe.
    \item Third Parties: entity that are interested to data provided by TrackMe.
    \item National Health Service: the national institution that provides health care to citizens.
    \item Social Security Number: a nine-digit-number that identifies uniquely a citizen.
    \item Fiscal Code: synonym to Social Security Number for Italian people.
    \item Credentials: user name, password, SSN or Fiscal Code.
    \item Anomaly: when the parameters are below a defined threshold. 
    \item Emergency: it occurs when an anomaly is detected.
    \item System: defines the overall set of software components that implement the required functionalities.
\end{itemize}

\subsection{Acronyms}
\begin{itemize}
    \item RASD: Requirements Analysis and Specification Document. 
    \item SSN: Social Security Number.
    \item GPS: Global Position System.
    \item API: Application Programming Interface.
    \item NHS: National Health Service.
\end{itemize}

\subsection{Abbreviation}
\begin{itemize}
    \item {[Gn]}: n-th goal
    \item {[Rn]}: n-th functional requirement
    \item {[Dn]}: n-th domain assumption
\end{itemize}

\section{Revision history}
Version 1.0 delivered on date 11/11/2018.

\section{Reference Documents}
\begin{itemize}
    \item Specification Document: "Mandatory Project Assignment AY 2018-19". 
    \item \href{https://ieeexplore.ieee.org/document/6146379} {29148-2011 - ISO/IEC/IEEE International Standard - Systems and software engineering -- Life cycle processes --Requirements engineering}
\end{itemize}

\section{Document Structure} %bozza di esempio, da riscrivere
\begin{enumerate} [label={Section \arabic*}]
    \item gives an introduction to the problem and describes the purpose of the applications Data4Help and Automated SOS and states their goals. The scope of the two applications is defined.
    \item presents the overall description of the project. The product perspective includes details on the shared phenomena and the domain models. The class diagram describes the domain model used,and the state diagram analyzes the process of arranging a meeting and reaching it in time. Here the majority of functions of the system are more precisely specified, with respect to the already mentioned goals of the system. In the user characteristics section the types of actors that can use the application are described.
    \item contains the external interface requirements, including: user interfaces, hardware interfaces, software interfaces and communication interfaces. Few scenarios describing specific situations are listed here. Furthermore, the functional requirements are defined by using use case and sequence diagram.
    The non-functional requirements are defined through performance requirements, design constraints and software system attributes.
    \item includes the alloy model and the discussion of its purpose. Also, a world generated by it is shown.
    \item shows the effort spent by each group member while working on this project.
    \item includes the reference documents.
\end{enumerate}