\section{Functional Requirements}
The design decisions illustrated in this document have been made with the purpose of satisfying all the functional requirements listed in the RASD document. In this section, each requirement is mapped to the corresponding design components.
The mapping can be better understood by observing the \hyperlink{RV}{\underline{Runtime view, section 2.5}}, in which is illustrated the sequence of the interactions between the design components, in order to accomplish the functional requirements.\\
Other important diagrams that integrate the following mapping, are in the\\ \hyperlink{CV}{\underline{Component view, section 2.3}}.

\begin{enumerate} [label={\bf[R\arabic*]}]
    \item \textbf{The system should provide a registration form offering the following mandatory fields: name, surname, SSN or Fiscal Code.}
        \begin{itemize}
            \item Mobile application
            \item User services: Login service
            \item Account manager
        \end{itemize}
    \item \textbf{The system must guarantee that the credentials are unique.}
        \begin{itemize}
            \item Account manager
        \end{itemize}
    \item \textbf{The system must acquire real-time users' positions from the GPS installed on the user's device.}
        \begin{itemize}
            \item Sensors API (GPS API)
            \item User services: Data acquisition service
            \item User services: Data sending service
            \item Central controller: Data gathering
        \end{itemize}
    \clearpage
    \item \textbf{The system must collect users' data in specific databases.}
        \begin{itemize}
            \item Central controller: Data gathering
            \item Central controller: Data provider
            \item Data manager
            \item DBMS
        \end{itemize}
    \item \textbf{The system must acquire real-time users' health parameters from the sensors installed on the user's device.}
        \begin{itemize}
            \item Sensors API
            \item User services: Data acquisition service
            \item User services: Data sending service
            \item Central controller: Data gathering
        \end{itemize}
    \item \textbf{Every time that a third party inserts a request for data relative to a Fiscal Code, the system must ask the authorization to the corresponding user.}
        \begin{itemize}
            \item Central Controller: Individual Request Manager
            \item User services: Request manager
            \item Mobile application
        \end{itemize}
    \item \textbf{The system must provide a function to allow users to accept or refuse a request for personal data.}
        \begin{itemize}
            \item Mobile application
            \item User services: Requests manager
        \end{itemize}
    \item \textbf{The system should provide a registration form to third parties.}
        \begin{itemize}
            \item Web application
            \item Mobile application
            \item Third party services: Registration service
        \end{itemize}
    \item \textbf{The system must provide a function to allow third parties to request the access to individuals data. The form must require the filling of the SSN or Fiscal Code of the corresponding user.}
        \begin{itemize}
            \item Web application
            \item Mobile application
            \item Third party services: Individual request service
            \item Account manager
            \item Central controller: Individual request manager
        \end{itemize}
    \item \textbf{The system must provide a function to allow third parties to request the access to data of a group of people. The form must offer some selection criteria such as geographical, time, health parameters, movement habits, ecc.}
        \begin{itemize}
            \item Web application
            \item Mobile application
            \item Third party services: Group request service
            \item Account manager
            \item Central controller: Group request manager
        \end{itemize}
    \item \textbf{As soon as a request has been approved, the system must forward the most recent data to the third party applying for them.}
        \begin{itemize}
            \item Web application
            \item Mobile application
            \item Third party services: Individual request service
            \item Third party services: Group request service
            \item Third party services: Access to data
            \item Central controller: Data provider
            \item Central controller: Data gathering
            \item Data Manager
            \item DBMS
        \end{itemize}
    \item \textbf{Data4Help must provide a function to let third parties download the received data.}
        \begin{itemize}
            \item Web application
            \item Mobile application
             \item Third party services: Access to data
        \end{itemize}
    \item \textbf{The system accepts a request for anonymous data only if the group composing the data is formed by more than 1000 people.}
        \begin{itemize}
            \item Central controller: Group request manager
        \end{itemize}
    \item \textbf{The system must provide a function that allows to customize the requests for subscription to updated data. The requests must include how long the permission is still valid.}
        \begin{itemize}
            \item Mobile application
            \item Web application
            \item Third party services: Individual request service
            \item Third party services: Group request service
        \end{itemize}
    \item \textbf{The system must provide a function to the corresponding user that allows him/her to accept or refuse the subscription request.}
        \begin{itemize}
            \item User services: Request manager
            \item Mobile application
        \end{itemize}
    \item \textbf{The system must forward in real-time the most recent data to the third parties that subscribed to them.}   
        \begin{itemize}
            \item Web application
            \item Mobile application
            \item Third party services: Individual request service
            \item Third party services: Group request service
            \item Third party services: Access to data
            \item Central controller: Data provider
            \item Central controller: Data gathering
            \item Data Manager
            \item DBMS
        \end{itemize}
\end{enumerate}

\section{Quality Requirements}
To respond to a huge number of \underline{concurrent} transactions we used the load balancing technique.
Data provided by the users and their access by several third parties could represent a huge amount of transactions that the system must be able to correctly handle.\\
The \underline{security} of the system is reached through at least four different mechanisms: the proxy, the Account Manager, the encryption of the data and the firewalls.\\
The proxy is able to detect DDoS attacks and, as a consequence, can disable a user that is overloading the servers.
This attacks are also prevented by the Account Manager, which guarantees that only the authorized users can interact with the servers.
All the databases are encrypted to prevent attacks that aims to steal sensitive data from the system.
Moreover, one of the purpose of the firewalls is to limit the number of ports that are exposed to the public.\\
The \underline{availability} and the \underline{reliability} of the system are ensured by the n-tier infrastructure. The system is based on several databases distributed on different data centers. The data are stored at least into two different databases so that when a server falls down, there is at least one available copy of them.\\
The \underline{maintainability} of the system is guaranteed by the Model View Controller design pattern and by the Facade pattern. \\
In order to ensure the \underline{portability} of the core service, the database and the back-end system are developed avoiding platform dependent programming languages. For this purpose the system is based on the Service Oriented Architecture.
