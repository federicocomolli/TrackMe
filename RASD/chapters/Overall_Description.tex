\section{Product perspective}
\section{Product functions}
    The following section lists the main functions offered by the system, focusing either on the end users or on the third parties.
    Of course, all the goals identified in the previous section will be offered as functions of the system-to-be.
    
    \subsection{Data4Help}
        \subsubsection {Data collection}
            The core function offered by Data4Help will be the real-time acquisition of the location and the health status of the users. The system will connect with the sensors' interface in the device OS and collect parameters such as GPS position, hearth rate, temperature, etc.
            All the data is permanently stored by the service and will be available to those who request them(*).
        
        \subsubsection {Data access to third parties}
            The system will consent to registered third parties to request an access to Data4Help data. This function can be divided in two categories: access to individual data and access to anonymized group of data. 
            Requests for individual data require to insert the SSN or Fiscal Code of the corresponding individual.
            The system will offer the possibility to customize a request for group data according to different criteria (age, gender, location, etc.).
            Third parties can also select the option of subscribing to the data, in order to receive them as soon as they are collected by Data4Help.
        
        \subsubsection {Accept or Refuse a request for data}
            When a third party requests the access to individual data, the system will notify the corresponding user and asks him/her to accept or decline the request. 
            The notification will show to the users the identity of the third party who requested their data.
        
    \subsection{AutomatedSOS}
        \subsubsection{User health monitoring}
            Using Data4Help technology, AutomatedSOS will be able to monitor in real-time the heath status of its registered users. 
            The system will compare the parameters provided by Data4Help with predefined thresholds to detect the current condition of the user.
        
        \subsubsection {Emergency intervention}
            If the system detects an anomaly in the user's parameters, an emergency request will be sent to the Operations Center.
            AutomatedSOS will compose a message containing all the user's personal information, his/her current location and the health parameters list.  
            The ambulance intervention won't be managed by the system, that can only guarantee to contact the NHS within 5 seconds from the anomaly detection.  

\section{User characteristics}
    In general, users of our system are not expected to be particularly tech-savvy. It is assumed that the users are comfortable to interact with a basic application on a mobile device.
    On the other side, third parties are expected to have a broader technological back-ground, since they will be interacting with the data sent by the service.
    In particular, third parties who requests data of a group of people has to know how to manage the huge amount of data received by the system.
    
    \subsection{Actors}
    \begin{itemize}
        \item Visitor: a person that is not registered yet. The only action he/she can perform is enter the registration process.
        \item User: a person who has registered to the service and, after the login phase, can exploit all the functionalities provided by the system.
        \item System Manager: an employee of TrackMe in charge of maintaining and updating the system.
        Possible updates are changes in the privacy criteria for group request and additions of new research categories in Data4Help, modifications in parameters' thresholds in AutomatedSOS.
        \item Third Party: an entity who has registered to the service using the apposite function to access data offered by the service.
        After the login phase it can request data of Data4Help users by filling the apposite form: it has to select some criteria for group request and it has to insert a Fiscal Code if it wants to receive the data of a single user. 
        \item Operations Center employee: the NHS operator working at the Operations Center at the moment in which an emergency request arrives from AutomatedSOS. This actor establishes the emergency level and coordinates the ambulance intervention.  
    \end{itemize}
    
\section{Assumptions, dependencies and constraints}

\subsection{Domain Assumptions}
Data4Help:  
\begin{enumerate} [label={[D\arabic*]}]
    \item Users insert credentials that correspond to their identity.
    \item When a new registered user sends his/her credentials to the system, the message will be surely received.
    \item The communication channel doesn't corrupt the data sent by the user's device to the system and vice versa.
    \item The electronic device on which the system is installed is equipped with the GPS sensor.
    \item The electronic device on which the system is installed is equipped with the sensors related to the parameters tracked by Data4Help.
    \item The system retrieves the data from the sensors through the interfaces available on the electronic device.
    \item Position information provided by GPS is sufficiently accurate\cite{gps}.
    \item The sensors related to the health status collect the data with a reasonable precision.
    \item It's sufficient that the number of people involved in a group query is greater than 1000 people to ensure the users' privacy.
\end{enumerate}  
\noindent
AutomatedSOS:
\begin{enumerate} [resume, label={[D\arabic*]}]
    \item Data provided by Data4Help API are surely received and they are not corrupted.
    \item When an emergency request is sent to National Health Service, it surely receive it.
    \item When National Health Service receives an emergency request, at least an ambulance is available.
    \item When parameters are below certain thresholds, it means that the user actually needs first aid.
    \item Operations Center of the NHS is up 24/7.
    \item When Operations Center gather a request from AutomatedSOS, it sends at least an ambulance and at least one of them arrives to the location of the emergency.
    \item NHS provides an API that allows third parties to send emergency requests in the form of a instant message received in their platform.
\end{enumerate}
